%%
%% INTRODUCTION
%%
\section{Introduction}

The introduction begins by introducing the field of research by contextualizing it and introducing main concepts. You will also highlighting some of the different currents and issues. Through this funneling process, you must arrive at your research topic, which will be described in the next section. One or more well-chosen examples to illustrate the issue and enable the reader to grasp its importance and interest could definitely help.

The purpose of the introduction is also to convince the reader that your work is worth reading. It should therefore be stressed that the problem is important and the contribution appropriate~\cite{melot2008elements}.
Finally, it's important to keep in mind that the reader doesn't necessarily know the subject and/or isn't an expert in the field~\cite{melot2008elements}.

Finally, the introduction is also the moment when the general plan of the manuscript is presented.

%%
%% PROBLEM STATEMENT
%%
\section{Problem Statement}

This part is mainly an adaptation of the problem state submitted for the January session.

\subsection{Research Field}

This section introduces the research field. First, it familiarizes the reader with the field by introducing the basic concepts and locating it in computer science. It then highlights the different trends and issues that make up research in the field.

\subsection{Research Topic}

You will begin by presenting your research topic more specifically. In addition to the presentation itself, you'll need to justify its relevance and interest in the context of your research field (why is it important?). This involves discussing existing solutions and their limitations~\cite{melot2008elements}.

\subsection{Problem Statement}

The research topic will then be formulated in the form of one or more research questions that define the scope of your work. These questions must, of course, be justified in terms of the subject and, more generally, the research area. But you also need to show that a literature review addressing those research questions is relevant.

%%
%% METHODOLOGY
%%
\section{Methodology}

The goal here is to explain how you are going to proceed to collect and analyze scientific literature to answer the research question(s).
In addition to a description of the methodology itself, which should make it possible to replicate the work carried out, the relevance of the choices made to the subject in question should be highlighted.
Finally, you must describe the conditions under which this work was carried out. If applicable, you should indicate whether and how you used artificial intelligence tools. This applies to article search, content extraction and analysis, and manuscript writing.

\subsection{Literature Research}

This section presents the bibliographical research methodology used to build up the corpus of references analyzed later.
In particular, we need to show that the methodology used has ensured that the references collected are representative of the research question.

The first step is to present the sources consulted, i.e., mainly bibliographic databases, and why they were chosen. 
Next, you'll need to present the search query(s) used, showing that they allow you to find a maximum number of relevant articles while limiting irrelevant references. To do this, you'll need to explain the different concepts manipulated and the keywords chosen.
Finally, where appropriate, other complementary methods can also be presented, explaining why they were used.

\subsection{Literature Selection}

This section describes and justifies the inclusion/exclusion criteria used for the selection of the relevant references. It also describes the selection process itself, in particular the number of references at the beginning and at the end.

\subsection{Extraction \& Analysis}

This section is dedicated to the tools and techniques used to extract and analyze information from your selection of references. You should show how you have extracted actionable information from a set of scientific articles.
With regard to extraction, you should present and justify the techniques used to extract from the references the information required for your state of the art (e.g., mindmap, analysis grid, comparative table, etc.).
The analysis part mainly describes the conceptual tools, existing or created by you, used to analyzing the selected references, why you use them, and how you applied them.

\subsection{Use of Generative AI Tools}

The use of generative AI tools (such as ChatGPT) is permitted in certain cases. However, you are responsible for the content created by these tools, and you have to describe how you used them in this section.

Generative AI tools may help you discover your subject (by “chatting” with it) and with bibliographic research (by suggesting keywords or by identifying sources of information). They can also be useful for summarizing or explaining articles, or even extracting specific information.
Finally, you can use them as brainstorming and ideation tools for the elaboration of the intellectual contribution (methodology, analysis, writing plan, etc.). In all cases, however, you should exercise caution, given the risk of confabulation.

Concerning the writing, the first authorized use is to improve the quality of the text (spelling, style, etc.). Simply mention the sections concerned and the tool(s) used in this section.
The second authorized case concerns content generation support (e.g., idea, figure, etc.). In addition to being mentioned in this section, the tool used must be explicitly cited in the relevant part of the text, in a similar way to a bibliographic citation\footnote{Formats and examples of how to cite generative AI tools can be found in \url{https://www.scribbr.com/ai-tools/chatgpt-citations/}}.

No other use is permitted, unless expressly authorized by Pr. Schobbens. In particular, you aren't allowed to use generative AI tools to generate complete sections of the final manuscript!

%%
%% STATE-OF-THE-ART
%%
\section{State-of-the-Art}

This section, whose structure is obviously specific to each subject, synthesizes the analysis of the selected references. In particular, this synthesis is written from the perspective of the research question. In any case, it's not enough simply to list selected references with brief summaries!
Finally, here, more than anywhere else, you need to be clear and concise in your writing.

%%
%% CONCLUSION
%%
\section{Conclusion}

The conclusion briefly reviews the work carried out, i.e., the review of the literature on a given subject and the main insights gained from it. In particular, it is an opportunity to provide an \textbf{answer} to the research question. It also identifies and discusses the \textbf{limitations} and the \textbf{perspectives} (often called \textit{Future work}) of the work carried out, which provide the basis for future research and in particular your master thesis (\textbf{mémoire}).
The conclusion is a very important part of the text~\cite{melot2008elements}. It summarizes the personal contribution of the work and highlights the main results.
